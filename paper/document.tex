%%This is a very basic article template.
%%There is just one section and two subsections.
\documentclass[article]{jss}

\usepackage{appendix}

%% almost as usual
\author{Jeroen Ooms\\UCLA Dept. of Statistics \And
        Second Author\\Plus Affiliation}
\title{The \pkg{RAppArmor} package: Linux Based Sandboxing of an \proglang{R}
Process in a Shared Environment}

%% for pretty printing and a nice hypersummary also set:
\Plainauthor{Jeroen Ooms, Second Author} %% comma-separated
\Plaintitle{The RAppArmor package: Linux Based Sandboxing of an R Process in a
Shared Environment} %% without formatting
\Shorttitle{The RAppArmor package} %% a short title (if necessary)


%% an abstract and keywords
\Abstract{
  With the increasing availability of public cloud computing facilities and
  scientific super computers, there is a great potential for making
  \proglang{R} available through public or shared resources. This allows
  researchers to efficiently run code requiring a lot of cycles and memory,
  or embed \proglang{R} functionality into e.g. systems or web services.
  However some important security concerns needs to be addressed before this
  can be put in production. The prime use case in the design of \proglang{R} has
  always been single statistician running \proglang{R} on the local machine
  through the interactive console. As a result there are practically no restrictions on
  what the user is allowed to do with the operating system, which could
  potentially result in malicious behavior or excessive use of hardware resources
  in a shared environment. Properly securing an \proglang{R} process turns
  out to be a complex problem. We describe some approaches of securing R
  and illustrate potential issues using some of our personal experiences in
  hosting public \proglang{R} services. Finally we introduce the \pkg{RAppArmor} package which
  provides a Linux based reference implementation for sandboxing \proglang{R}
  on the level of the operating system. }


\Keywords{R, Security, Linux, Sandbox, AppArmor}
\Plainkeywords{R, security, linux, sandbox, apparmor}

%% publication information
%% NOTE: Typically, this can be left commented and will be filled out by the technical editor
%% \Volume{13}
%% \Issue{9}
%% \Month{September}
%% \Year{2004}
%% \Submitdate{2004-09-29}
%% \Acceptdate{2004-09-29}

%% The address of (at least) one author should be given
%% in the following format:
\Address{
  Jeroen Ooms\\
  UCLA Department of Statistics\\
  University of California\\
  E-mail: \email{jeroen.ooms@stat.ucla.edu}\\
  URL: \url{http://www.stat.ucla.edu/~jeroen}
}

\begin{document}

\section{Security in R: Introduction and Motivation}

The \proglang{R} project for statistical computing and graphics
\citep{R-project} is currently one of the primary tool-kits for scientific
computing. The software is widely used for research and data analysis in both
academia and industry, and is the de-facto standard among statisticians for the
development of new statistical computation methods. With support for all major
operating systems, a powerful and stable codebase, more than 3000 addon
packages and a huge active community, it is fair to say that the project has
matured to a production-ready computation tool. However, one thing that is
somewhat surprising is that the way in which \proglang{R} is used, has hardly
changed since its initial design. Even though internet access, public
cloud computing, live and open data and scientific super computers have transformed
the landscape of data analysis, \proglang{R} is still almost exclusively used
as an end-user tool, running on the local machine of the researcher, operated
through the interactive console. This seems somewhat of a missed opportunity.
The demand for data analysis tools has never been higher, and many open source
systems and software stacks could benefit greatly from the high quality
analytical capabilities that \proglang{R} has to offer.

One reason developers are reluctant to build on R are concerns regarding
security and management of shared hardware resources. Reliable software systems
require components which behave predictably and cannot be abused. Because
\proglang{R} was primarily designed with in mind a single user running the
software on the local machine through the interactive console, security issues
and unpredictable behavior have not been considered a major concern in the
design of \proglang{R} itself. Hence, these problems will need to be addressed
somehow before software developers will feel comfortable making \proglang{R}
part of their infrastructure, or convince administrators to use their
facilities to expose \proglang{R} based services to the public. It is our
personal experience that the complexity of these issues is easily
underestimated when designing stacks or systems that build on \proglang{R}.
There are a number of issues that are very domain specific to scientific
computing, which makes building on \proglang{R} quite different from embedding
other software or languages. This might explain the limited adoption of
\proglang{R} as a computational back-end engine so far.

\subsection{Security when using Contributed Code}

Building systems on \proglang{R} has been the main motivation for this research.
However, security is becoming a concern for \proglang{R} in other contexts as
well. As the \proglang{R} community is growing rapidly, it becomes more unsafe
to rely on the social aspect of contributed code. For example, on a daily
basis, dozens of packages and package updates submitted to CRAN. These packages
can contain code written in R, C, Fortran, C++, Java, etc. It is practically
impossible for the CRAN maintainers to do a thorough audit of the full code
that is submitted, every time. Some packages even contain pre-compiled Java code
for which the source is not included at all. Furthermore, \proglang{R} packages
are not signed with a private key as is the case for e.g. packages in most Linux
distributions, which makes it hard to verify the identity of the author. As
CRAN packages are automatically build and installed on hundreds, maybe thousands
of machines around the world, they become a more interesting target for people
with questionable motives. Hence there is a real risk of packages containing
malicious code making their way unnoticed into CRAN. Risks are even greater for
packages that are distributed through channels without any form of code review,
for example via email or through the increasingly popular Github repositories
\citep{torvalds2006git,dabbish2012social}.

In summary, it is not unreasonable for the \proglang{R} user to be a bit
cautious when installing and running contributed code downloaded from the
internet. However, things don't have to be as serious as described above.
Thinking about security is a good practice, even if there is no immediate cause
for concern. Some users simply might want to protect against themselves, making
sure they don't erase any files by accident and that \proglang{R} does not
interfere with other activities on the same machine. Knowing that \proglang{R}
is running with no unnecessary privileges can be reassuring to the user and
system administrators, and might one day prevent a lot of trouble. 

\subsection{Sandboxing the R environment}

This paper explores some of the potential problems, along with approaches and
methods of securing \proglang{R}. Some of the different aspects and concerns of
security in the context of \proglang{R} are illustrated using personal
experiences, or examples of bad or malicious code. We will explain how untrusted
code can be run inside a \emph{sandboxed} \proglang{R} process. Sandboxing in
this context is a somewhat informal term for creating an environment in which
untrusted software runs without capabilities of doing anything harmful to the
system. As it turns out, \proglang{R} itself is not very suitable for managing
access control policies, and the only way to enforce security properly is by
leveraging features from the operating system. To exemplify this approach, a
reference implementation based on Linux and AppArmor is provided which can be
used on Linux distributions as the basis for a sandboxing toolkit. This package
is used throughout the paper to show some of the issues could be addressed.

However, we want to emphasize that we don't claim to have solved the problem.
This paper mostly serves an introduction to security for the \proglang{R}
community, and hopefully creates some awareness that this is a real issue moving
forward. The \pkg{RAppArmor} package is one approach and a good starting point
for experimenting with dynamic sandboxing in \proglang{R}. However it mostly
serves as a proof of concept of the general idea confining and controlling \proglang{R}
in order to extend the applicability. The paper describes some examples of use
cases, issues, policies and personal experiences which give the reader a sense
of what is involved with this topic. Without any doubt, there are other concerns
beyond the ones mentioned in this paper, many of which might be specific to
certain applications or systems. We hope to invoke a discussion in the community
about potential security threads and solutions related to using \proglang{R} in
different scenarios, and encourage those comfortable with other platforms or use
\proglang{R} in different contexts to join the discussion and share their
concerns, experiences and solutions.
 

\section{Use-Cases and Concerns of Sandboxing R}

Let us start by taking a step back and put this research in perspective by
describing some example use cases where security in \proglang{R} could be a
concern. Below 3 simple examples of situations in which it is useful to be able
to run \proglang{R} inside a sandbox. The uses cases are increasing in
complexity, and require more advanced sandboxing methods. 

\subsubsection{Running Untrusted Code}

Suppose we found an \proglang{R} package in our email or on the internet that
looks interesting, but we are not quite sure who the author is, and if the
package does not contain any malicious code. The package is too large for us to
inspect all of the code manually, and furthermore it contains a library in a
foreign language (e.g. \proglang{C++}, \proglang{Fortran}) for which we lack
knowledge to really assess what exactly is going on. We would like to give the
package a try, but without exposing ourselves to the risk of potentially
jeopardizing the machine.

One solution would be to run the code on a separate or virtual machine. However
this is somewhat cumbersome and we will not have our regular workflow available.
In practice creating new machines is a bit unpractical and not something that we
might want to do on a daily basis. It would be easier if we could just sandbox
our regular \proglang{R} environment for the duration of installing and using
the new package. If the sandbox flexible and unobtrusive enough to not interrupt
our daily workflow, we could even make a habit out of using it every time we use
contributed code (which for most users is every day).


\subsubsection{Shared Resources}

Another use case could be a scenario where multiple users are sharing a single
machine. For example, a system administrator at a university is managing a big
computing resource and would like to make it available to faculty and students
for using \proglang{R}. This way they could run \proglang{R} code that requires
more computing power than their local machine can handle. For example a
researcher might want to do a simulation study, and fit a complex model a
million times on generated datasets of varying properties. On her own machine
this would take months to complete, but the super computer can finish the job
overnight. The administrator would like to give this and other researchers an
automated way to run their \proglang{R} code on the supercomputer. However he is
concerned about users interfering with each others work, or breaking anything
on the machine. Furthermore he wants to make sure that system resources are
allocated in a fair way so that no single user can consume all memory or cpu
on the system.

\subsubsection{Embedded Systems and Services}

There have been a number of efforts to facilitate integration of \proglang{R}
functionality into various 3rd party systems. Some examples of interfaces from
popular general purpose languages are \texttt{RInside} \citep{RInside}, which
embeds \proglang{R} into C++ environments, and JRI/REngine \texttt{JRI} which
embeds R in Java software. Similarly, rpy \citep{moreira2006rpy} provides a
Python interface to R, and RinRuby is a Ruby library that integrates the
\proglang{R} interpreter in Ruby \citep{dahl2008rinruby}. Littler provides
hash-bang (i.e. script starting with \texttt{\#!/some/path}) capability for
\texttt{R} \citep{littler}. The Apache2 module RApache  (\texttt{mod\_R})
\citep{rapache} makes it possible to run \texttt{R} scripts from within the
Apache2 web server. \cite{heiberger2009r} provide a series of tools to call
\proglang{R} from DCOM clients on Windows environments, mostly to support
calling \proglang{R} from Microsoft Excel. Finally, \texttt{RServe} is TCP/IP
server which provides low level access to an \proglang{R} session over a socket
\citep{Rserve}.

The third use case originates from these developments: it can be summarized as
confining and managing \proglang{R} processes inside of embedded systems and
services. This use case is largely derived from our personal experience: we are
using \proglang{R} inside a number of systems and web services that provide
on-demand calculations and plotting over the internet. These services have to
respond quickly and with minimal overhead to incoming requests, and should
scale to serve many jobs per second. Furthermore the systems have to be stable,
requiring that jobs should always return within a given timeframe. Depending on
user and the type of job, different security restrictions might be appropriate.
Also we need a way to dynamically enforce limits on the use of memory,
processors and disk space on a per process basis. These requirements demands
a more flexible and finer degree of control over the process privileges and
restrictions than the first two use cases. It encouraged us to explore more
advanced methods than the conventional tools and forms the most central
motivation of this research.

\subsection{System Privileges and Hardware Resources}
 
The use cases described above provide motivations and requirements for an
\proglang{R} sandbox. Two inter-related problems can be distinguished. The first
one is preventing system abuse, i.e. use of the machine for malicious or
undesired activities, or completely compromising the machine. The second
problem is managing sharing of hardware resources, i.e. preventing excessive
use of resources by limiting the amount of memory, cpu, etc that a single user
or process is allowed to consume.

\subsubsection{System Abuse}

The \proglang{R} console gives the user direct access to the operating system
and does not implement any privileges restrictions or access control to prevent
malicious use. In fact, some of the basic functionality in \proglang{R} actually
assumes quite profound access to the system, e.g. read access to system files,
or the privilege of running system shell commands. Therefore, running user
supplied \proglang{R} code without any restrictions can get us in serious
trouble. For example, the code could call the \texttt{system()} function which
provides an interface to the system shell. From here any system commands can be
executed, which can potentially be harmful. But also innocent looking functions
like \texttt{read.table} can be used to extract sensitive information from the
system, e.g. \texttt{read.table("/etc/passwd")} will gives us a list of users
on the system or \texttt{readLines("/var/log/syslog")} shows system log
information.

Even an \proglang{R} process running as a non-privileged user can do a lot of
harm. Some potential issues are code that contains or downloads a virus or
security exploit, or searches the system for the users personal information.
Appendix \ref{creditcard} shows an example of a simple function that searches
the users home directory for documents containing credit card numbers. Another
increasing global problem are viruses that make the machine part of a so called
botnet \citep{abu2006multifaceted}. Once infected, the compromised machines
(``bots'') connect to a centralized server and wait for instructions from the
owner of the botnet. Botnets are mostly used to send spam or to participate in
DDOS attacks: centrally coordinated operations in which a large number of
machines on the internet is used to flood a server or provider with network
traffic with the goal of taking it down by overloading it
\citep{mirkovic2004taxonomy}. Botnet software is often invisible to the user
of an infected machine and can run with very little privileges: just network
access is sufficient to do most of the work.

When using \proglang{R} on the local machine and only running our own code, or
from trusted sources, these scenarios might sound a bit far fetched. However,
when running code downloaded from the internet or exposing systems to the
public, this is becoming a real concern. Internet security is a global problem,
and there are a large number of individuals, organizations and even governments
actively employing increasingly advanced and creative ways of gaining access to
others machines. Especially servers that run on beefy hardware or fast
connections are attractive targets for individuals that could use these
resources for other purposes. But also servers and users inside large companies,
universities or government institutions are frequently targeted with the goal
of gathering confidential knowledge. This last aspect seems especially
relevant, as \proglang{R} is used frequently in these organizations.

\subsubsection{Resource Restrictions}

The other category of problems is not so much related to intentional abuse, and
might even arise unintentionally. It is fair to say that \proglang{R} can be
quite greedy with hardware resources. One can easily run a command which will
consume all of the available memory and/or CPU, and does not finish executing
unless manually terminated. When running \proglang{R} on the local machine
through the interactive console, the user will quickly recognize a function
call that is not returning timely, and can interrupt the process prematurely by
sending a \texttt{SIGINT} by pressing \texttt{CTRL+C} in Linux or \texttt{ESC}
in Windows. If this doesn't work we can open the task manager and tell the
operating system to kill the process.

However, when \proglang{R} is embedded in a bigger system, things are more
complicated, and we have to think about these scenarios in advance. When an
out-of-control R job is not properly detected, the process might very well run
forever and take down our service, or even the entire machine. This has actually
been a major problem that we personally experienced in an early implementation
of a public web service for mixed modelling \citep{yeroonlme4} which uses the
\texttt{lme4} package \citep{lme4}. What happened was that users could
accidentally specify a variable with many levels as the \emph{grouping factor}
which would cause the design matrix to blow up, even on a relatively small
dataset, and decompositions would take forever to complete. To make things
worse, \texttt{lme4} uses a lot of \texttt{C} code which does not respond do
time limits set by R's \texttt{setTimeLimit} function. This happened multiple
times, and the only way to get things up again was to manually login to the
server and reset the application.

This example is not an exception. The behavior of \proglang{R} can sometimes be
unpredictable, which is an aspect that is easily overlooked by
(non-statistician) developers. When a system calls out to e.g. an \texttt{SQL}
or \texttt{php} script, the script usually runs without any problems and the
time needed to process is proportional to the size of the data, i.e. the number
of returned records returned by \texttt{SQL}. However, in an \proglang{R}
script, many things can go wrong, even though the script itself is perfectly
fine. Algorithms might not converge, data might be rank-deficient, or missing
values throw a spanner in the works. Even when we only use tested code or
predefined services, this does not always entirely guarantee smooth and timely
completion of \proglang{R} jobs. When using \proglang{R} in systems or shared
facilities, it is important that we take this aspect into account and have a
way of dealing with this that does not require manual intervention.

\section{Different Approaches of Restricting R}

The current section introduces some approaches of securing and sandboxing
\proglang{R}, with their advantages and limitations. They are reviewed in the
context of our use cases, and evaluated on how they address the problems of
system abuse and resource restrictions. The approaches are increasingly
\emph{low-level}: they represent security on the level of the application, R
software itself and operating system. As will become clear, we are leaning
towards the opinion that \proglang{R} itself is not very well suited to address
security issues, and the only way to do proper sandboxing is on the level of the
operating system. This will lead us to the introduction of the \pkg{RAppArmor}
package, which is described in the next section.


\subsection{Application Level Security: Predefined Services}

The most common approach to preventing system abuse is simply to only allow a
limited set of predefined services, that have been deployed by a trusted
developer and cannot be abused. This is generally the case in websites
containing dynamic content though e.g. CGI or PHP scripts. Running arbitrary
code is explicitly prevented and any possibility to do so anyway is considered
a security hole. For example, we might want to expose the following function as
a web service:

\begin{CodeChunk}
\begin{CodeInput}
liveplot <- function (ticker) {
  url <- paste("http://ichart.finance.yahoo.com/table.csv?s=",
    ticker, "&a=07&b=19&c=2004&d=07&e=13&f=2020&g=d&ignore=.csv",
    sep = "")
  mydata <- read.csv(url)
  mydata$Date <- as.Date(mydata$Date)
  myplot <- ggplot2::qplot(Date, Close, data = mydata, geom = c("line",
    "smooth"), main = ticker)
  print(myplot)
}
\end{CodeInput}
\end{CodeChunk}

This function above downloads live data from the public API at Yahoo Finance and
creates an on-demand plot of the historical values using \texttt{ggplot2}
\citep{ggplot2}. The function has only one parameter, \texttt{ticker}, which is
a character string identifying a stock symbol. This function can be exposed as a
predefined web service, where the client only supplies the \texttt{ticker}
argument. Hence the system does not need to run any potentially harmful
user-supplied \proglang{R} code. The client can only set the symbol to e.g.
\texttt{'GOOG'} and the resulting plot can be returned in the form of a
PNG image or PDF document. This function is actually the basis of the
``stockplot'' web application \citep{stockplot}; an interactive graphical web
application for financial analysis which still runs today.

Limiting users or clients to execute only predefined services is often the
easiest solution, but rather limited in application and actually not 100\%
secure. A predefined service can be nice to do some canned calculations or
generate a plot as done in the example, but for most \proglang{R} applications
it quickly turns out to be overly restrictive. For example in case of an
application that allows the user to fit a statistical model, the user might
need to be able to include transformations of variables like \texttt{I(cos(x\^\ 2))} or \texttt{cs(x, 3)}. Not allowing a
user to call any custom functions makes this hard to implement.
Furthermore, when using only predefined services, all the work and
responsibility is put in the hands of the developer and administrator. Only they
can expose new services and they have to make sure that all services that are
exposed cannot be abused in some way or another. Therefore this approach is
expensive, and not very social in terms of users contributing additional
services. In practice, anyone that wants to publish an \proglang{R} service
will have to purchase and manage a personal server or know someone that is
willing to do so.

Also it might still be necessary to set hardware limitations, even when exposing
relatively simple, restricted services. We already mentioned the example of the
\texttt{lme4} web application, where a single user could accidentally take down
the entire system by specifying an overly complex model. Hence, restricting to
predefined services does not quite guarantee smooth and timely completion of
\proglang{R} jobs.

\subsubsection{Code Injection}

Finally, there is still the risk of \emph{code injection}. Because \proglang{R}
is a very dynamic language, evaluations sometimes happen at unexpected places.
One example is during the parsing of \texttt{formulas}. For example, we might
want to publish a service that calls the \texttt{lm()} function in \proglang{R}
on a fixed dataset. Hence the only thing the user can supply is a \emph{formula}
in the form of a character vector. Assume in the code snippet below that
the \texttt{userformula} is a string that has been supplied by a user through
some graphical interface.

\begin{CodeChunk}
\begin{CodeInput}
glm(userformula, data=cars)
\end{CodeInput}
\end{CodeChunk}

For example the user might supply a string \texttt{"speed
{\raise.17ex\hbox{$\scriptstyle\sim$}} dist"} and the service will return the
coefficients. On first sight, this might seem like a safe service. However,
formulas actually allow for the inclusion of calls to other functions. So even though the \texttt{userformula} is a character vector, we can actually use it to inject a
function call:

\begin{CodeChunk}
\begin{CodeInput}
userformula <- "speed ~ dist + system('whoami')"
lm(userformula, data=cars)
\end{CodeInput}
\end{CodeChunk}

In the example above, \texttt{lm} will automatically convert
\texttt{userformula} from type character to a \texttt{formula}, and
subsequently execute the \texttt{system("whoami")} command. So even when a user
can supply only very simple primitive data, it is still important to sanitize
the input before calling the service. One way to do so is to set up the service
in such a way that only alphanumeric values are needed for the parameters, and
use a regular expression to remove any other characters, before actually
executing the script or service:

\begin{CodeChunk}
\begin{CodeInput}
myarg <- gsub("[^a-zA-Z1-9]", "", myarg)
\end{CodeInput}
\end{CodeChunk}


\subsection{Sanitizing and Blacklisting}

A less restrictive approach is to allow users to push custom R code, but inspect
the code before evaluating it to make sure it does not contain malicious calls.
This approach has been adopted with some web sites that allow users to run
\proglang{R} code, like \cite{banfield1999rweb} and \cite{cloudstat}. However,
given the dynamic nature of \proglang{R}, this is actually very hard to do and
is often easy to circumvent. For example, one might want to prevent users from
calling the \texttt{system} function. One way is to define some smart regular
expressions that look for the word ``system'' in a block of code. This way
it would be possible to detect a potentially malicious call like this:

\begin{CodeChunk}
\begin{CodeInput}
system("whoami")
\end{CodeInput}
\end{CodeChunk}

However, it will be much harder to detect the equivalent call in the following
block:

\begin{CodeChunk}
\begin{CodeInput}
foo <- get(paste("sy", "em", sep="st"))
bar <- paste("who", "i", sep="am")
foo(bar)
\end{CodeInput}
\end{CodeChunk}

And indeed, it turns out that the services that use this approach are fairly
easy to hack. Because \proglang{R} is a dynamic scripting language, the exact
function calls might not reveal themselves until runtime, when it is often too
late. We are actually quite convinced that it is nearly impossible to really
sanitize an \proglang{R} script just by inspecting the source code.

An alternative method to do sanitizing is to define an extensive whitelist of
functions that a user is allowed to call, and mask all other functions. The
\pkg{sandboxR} \citep{sandboxR} package uses this method to block access
to all R functions that provide access to the file system. It evaluates
the user-supplied code in an environment in which all blacklisted
functions are masked from the calling namespace. This is fairly effective and
can be useful for some applications. However, the method relies on exactly
knowing and specifying which functions are \emph{safe} and which are not. The
package author has done this for the thousands of \proglang{R} functions in the
base package and we assume he has done a good job. However, it makes it hard to
maintain and cumbersome to generalize the approach to other R packages (by
default the method does not allow loading other packages). Furthermore the
entire method falls if there is one function that has been
overlooked, which does make the method somewhat vulnerable.

Moreover, even when sanitizing of the code is successful, this method does not
limit the use of hardware resources in any way. Hence, additional methods are
still required to prevent excessive use of resources in a public environment.

\subsection{Sandboxing on the Level of the Opering System}

One can argue that managing hardware and security privileges is something that
is outside the domain of the \proglang{R} software, and is better let to the
operating system. The \proglang{R} software has been designed for statistical
computing and related functionality; the operating system deals with hardware
and security related matters. Hence, in order to really sandbox \proglang{R}
properly without imposing unnecessary limitations on its functionality, we need
to sandbox the \emph{R-process} on a lower level in the OS. When restrictions
are enforced by the operating system instead of \proglang{R} itself, we do not
have to worry about all of the pitfalls and implementation details of
\proglang{R}. The user can interact freely with \proglang{R}, but won't be able
to do anything for which the system does not grant permissions. Unfortunately,
this approach comes at the cost of portability of the software. Different
operating systems implement very different methods for managing processes and
privileges, so the solutions will be to a large extend OS-specific. However we
can still create interfaces from \proglang{R} to interact directly with the
operating system. And as is often the case in \proglang{R}, eventually these
functions can behave somewhat OS specific, providing similar functionality on
different systems and abstract away low level technicalities.

Some operating systems offer more advanced capabilities for setting process
restrictions than others. The most advanced functionality is found in
\texttt{UNIX} like systems, of which the most popular ones are either
\texttt{BSD} based (\texttt{FreeBSD, OSX,} etc) or \texttt{Linux} Based
(\texttt{Debian, Ubuntu, Fedora, Redhat, Suse,} etc). Most \texttt{UNIX} like
systems implement some sort of \texttt{ULIMIT} functionality to facilitate
restricting availability of hardware resources on a per-process basis.
Furthermore, on both \texttt{BSD} and \texttt{Linux} there are a number of
\emph{Mandatory Access Control} (\texttt{MAC}) systems available. On Linux,
these are offered as Kernel modules. The most popular ones are \texttt{AppArmor}
\citep{apparmor}, \texttt{SELinux} \citep{selinux} and \texttt{Tomoyo Linux}
\citep{tomoyo}. MAC provides a much finer degree of control than standard
user-based privileges, by applying advanced security policies on a per-process
basis.

Using a combination of \texttt{MAC} and \texttt{ULIMIT} tools we can do a pretty
decent job in sandboxing a single R process to a point where it can do little
harm to the system. Hence we can run arbitrary \proglang{R} code without losing
any sleep over potentially jeopardizing the machine. In the next section a
reference implementation is presented based on the \texttt{Linux} and
\texttt{AppArmor}.

\section{the RAppArmor package}

The current section describes some security concepts and how an R process can be
sandboxed using a combination of \texttt{ULIMIT} and \texttt{MAC} tools. The methods
are illustrated using the \pkg{RAppArmor} package: an implementation based on
\texttt{Linux} and \texttt{AppArmor}. AppArmor ("Application Armor") is a
security module for the Linux kernel. It allows the system administrator to
associate \emph{security profiles} to programs and processes that restrict the
capabilities and permissions of that process. The \pkg{RAppArmor} \proglang{R}
package implements some convenient \proglang{R} interfaces to Linux system
calls related to setting privileges and restrictions of a running process.
Besides applying AppArmor profiles, \pkg{RAppArmor} also interfaces to
the Linux \texttt{prlimit} call, which sets \texttt{RLIMIT} (resource limit) values on a
process (\texttt{RLIMIT} are the linux implementation of \texttt{ULIMIT}).
Linux defines a number of \texttt{RLIMIT}'s, which define resources like
memory, number of processes, and stack size. More on \texttt{RLIMIT} later in
section \ref{RLIMITS}.

There are two ways of using AppArmor. One is to automatically associate a
single security profile with every \proglang{R} process. This can also be done
without \pkg{RAppArmor}. However, this is often somewhat limited and overly
restrictive. Using the RAppArmor package, \proglang{R} commands can be
dynamically evaluated in a secured sandbox with a custom uid, security profile and hardware restrictions,
with minimal overhead. This gives the user the freedom to design code in which
only the insecure parts are sandboxed, which turns out to be quite powerful.

\subsection{AppArmor Profiles}

The security profiles are the core of the AppArmor software. A profile is
defined by a set of rules in a text file using AppArmor syntax. The Linux
kernel translates these rules to a security policy that it will enforce on the
appropriate process. A brief introduction to the AppArmor syntax is given in
section \ref{syntax}. The appendix of this paper contains some example
profiles that ship with the \pkg{RAppArmor} package to get the user started.
When the package is installed through the Ubuntu installer (e.g. using apt-get)
the profiles are automatically copied to \texttt{/etc/apparmor.d/rapparmor.d/}.
Because profiles define file access permissions based the location of files and
directories on the file system, they are to some extend specific to a certain
Linux distribution, as different distributions have somewhat varying conventions
on where files are located. The example profiles included with \pkg{RAppArmor}
are based on the file layout of the \pkg{r-base} package (and its dependencies)
by \cite{batesusing} for Debian/Ubuntu, currently maintained by Dirk Eddelbuettel.

The \texttt{RAppArmor} package and the included profiles work ``out of the
box'' on Ubuntu 12.04 (Precise) and up. Also it should be working on Debian
7.0 (Wheezy) and up, however as of writing of this paper, this distribution is
still in the ``testing'' phase. Furthermore the package has been successfully
build on OpenSuse 12.1. Note that Suse systems organize the file system in a
slightly different way than Ubuntu and Debian, so the profiles should be
modified accordingly. 

Again, we want to emphasize that the package and included
profiles should mostly be seen as a \emph{reference implementation}. Using the
package we demonstrate how to create a working sandbox using AppArmor. However,
depending on system and application, different policies might be appropriate.
The \pkg{RAppArmor} package provides the tools to set security restrictions in
\proglang{R} and ships with some example profiles to get the user started.
However it is still up to the administrator to determine which security
policy is appropriate for a certain system and context. The example
profiles are a good starting point, but should be fine-tuned for specific
applications.

\subsection{Automatic Installation}

The \pkg{RAppArmor} package consists of an R package and a number of security
profiles. On Ubuntu Linux the package is most easily installed using the binary
packages provided through launchpad:

\begin{CodeChunk}
\begin{CodeInput}
sudo add-apt-repository ppa:opencpu/rapparmor
sudo apt-get update
sudo apt-get install r-cran-rapparmor
\end{CodeInput}
\end{CodeChunk}

One can also create the Ubuntu packages from the source \proglang{R} package
using something along the lines of the following:

\begin{CodeChunk}
\begin{CodeInput}
tar xzvf RAppArmor_0.4.0.tar.gz
cd RAppArmor/
debuild -uc -us
cd ..
sudo dpkg -i r-cran-rapparmor_0.4.0-precise4_amd64.deb
\end{CodeInput}
\end{CodeChunk}

The \texttt{r-cran-rapparmor} Ubuntu package will automatically install required
dependencies and security profiles. The security profiles are installed in
\texttt{/etc/appamor.d/rapparmor.d/}. 

\subsection{Manual Installation}

To install the package on a distribution for which no installation package is
available, one might need a manual installation. To build the package manually
several steps are needed. First of all, one needs to make sure the required
dependencies are installed:

\begin{CodeChunk}
\begin{CodeInput}
sudo apt-get install r-base-dev libapparmor-dev apparmor
\end{CodeInput}
\end{CodeChunk}

Note that the package requires \proglang{R} version 2.14 or higher. Also the
system needs to have an apparmor enabled Linux kernel. After these packages are
installed, one can proceed installing \pkg{RAppArmor} in \proglang{R}, using
e.g:

\begin{CodeChunk}
\begin{CodeInput}
sudo R CMD INSTALL RAppArmor_0.4.0.tar.gz
\end{CodeInput}
\end{CodeChunk}

This will compile the \proglang{C} code and install the \proglang{R}
package. After the package has been installed successfully, the security
profiles need to be copied to the \texttt{apparmor.d} directory:

\begin{CodeChunk}
\begin{CodeInput}
sudo cp -Rf /usr/local/lib/R/site-library/RAppArmor/profiles/debian/*
/etc/apparmor.d/
\end{CodeInput}
\end{CodeChunk}

Finally, the apparmor service needs to be restarted to load the new profiles.
Also we do not want to enforce default the R profiles at this point yet:

\begin{CodeChunk}
\begin{CodeInput}
sudo service apparmor restart
sudo aa-disable usr.bin.r
\end{CodeInput}
\end{CodeChunk}

This should complete the installation. To verify if everything is working, start 
\proglang{R} and run the following code:

\begin{CodeChunk}
\begin{CodeInput}
library(RAppArmor)
aa_change_profile("r-base")
\end{CodeInput}
\end{CodeChunk}

If the code runs without any errors, the package has successfully been
installed.

\subsection{Linux Security Methods}

The \pkg{RAppArmor} package interfaces to a number of Linux system calls that
are useful in the context of security and sandboxing. The advantage of calling
these directly from \proglang{R} is that we can dynamically set the parameters
from within the \proglang{R} process, as opposed to fixing them for every
\proglang{R} session. Hence it is actually possible to execute some parts of an
application in a different security context other parts, which can be useful in
large applications.

The package defines a lot of low level functions that wrap around Linux
\proglang{C} interfaces. However it is not required to study all of these
functions. For the end user, everything in the package comes together in the
powerful and convenient \texttt{eval.secure()} function. This function mimics
\texttt{eval()}, but it has additional parameters that define restrictions
which will be enforced to this specific evaluation. For example, one could use

\begin{CodeChunk}
\begin{CodeInput}
myresult <- eval.secure(myfun(), RLIMIT_AS = 10*1024*1024, profile="r-base")
\end{CodeInput}
\end{CodeChunk}

Which will call \texttt{myfun()} with a memory limit of 10MB and the ``r-base''
security profile (which is introduced in section \ref{r-base}). The
\texttt{eval.secure} function works by creating a \emph{fork} of the current process, and then set hard limits, UID and apparmor profile on the forked process, before evaluating the call. After the function
returns, or when the timeout is reached, the forked process is killed and
cleaned up. This way, all of the one-way security restrictions can be applied,
and evaluations that happen inside \texttt{eval.secure} won't have
any side effects on the main process.

\subsection{Setting User and Group ID}

One of the most basic security methods is running a process as a specific user.
Especially within a system where the main process has superuser privileges
(which could be the case in for example a webserver), switching to a user with
limited privileges before evaluating any code is a wise thing to do. We could
even consider a design where every user of the application has a dedicated
user account on the Linux machine. The \pkg{RAppArmor} package implements the
functions \texttt{getuid, setuid, getgid, setgid}, which call out to the
respective \texttt{Linux} system calls. Users and groups are defined as integer
values as specified inside the \texttt{/etc/passwd} file.

\begin{CodeChunk}
\begin{CodeInput}
> library(RAppArmor)
> system('whoami')
root
> getuid()
[1] 0
> getgid()
[1] 0
> setgid(1000)
Setting gid...
> setuid(1000)
Setting uid...
> getgid()
[1] 1000
> getuid()
[1] 1000
> system('whoami')
jeroen
\end{CodeInput}
\end{CodeChunk}

The user/group ID can also be set inside the \texttt{eval.secure} function. In
this case it will not affect the main process; the UID is only set for the time
of the secure evaluation.

\begin{CodeChunk}
\begin{CodeInput}
> eval(system('whoami', intern=TRUE))
[1] "root"
> eval.secure(system('whoami', intern=TRUE), uid=1000)
[1] "jeroen"
> eval(system('whoami', intern=TRUE))
[1] "root"
\end{CodeInput}
\end{CodeChunk}

Note that in order for \texttt{setgid} and \texttt{setuid} to work, the user
must have the appropriate capabilities in \texttt{Linux}, which are usually
restricted to users with superuser privileges. The \texttt{getuid} and
\texttt{getgid} functions can be called by anyone.

\subsection{Linux Resource Limits (RLIMIT)}
\label{RLIMITS}

Linux defines a number of \texttt{RLIMIT} values that can be used to set
resource limits on a process \citep{linuxrlimit}. The \pkg{RAppArmor} package
has functions to get/set to the following RLIMITs:

\begin{itemize}
  \item \texttt{RLIMIT\_AS} -- The maximum size of the process's virtual memory
  (address space).
  \item \texttt{RLIMIT\_CORE} -- Maximum size of core file.
  \item \texttt{RLIMIT\_CPU} -- CPU time limit.
  \item \texttt{RLIMIT\_DATA} --  The maximum size of the process's data
  segment.
  \item \texttt{RLIMIT\_FSIZE} --  The maximum size of files that the process
  may create.
  \item \texttt{RLIMIT\_MEMLOCK} -- Number of memory that may be locked into
  RAM.
  \item \texttt{RLIMIT\_MSGQUEUE} -- Max number of bytes that can be allocated
  for POSIX message queues
  \item \texttt{RLIMIT\_NICE} --  Specifies a ceiling to which the process's
  nice value (priority).
  \item \texttt{RLIMIT\_NOFILE} -- Limit maximum file descriptor number that can
  be opened.
  \item \texttt{RLIMIT\_NPROC} -- Maximum number of processes (or, more
  precisely on Linux, threads) that can be created by the user of the calling process.
  \item \texttt{RLIMIT\_RTPRIO} -- Ceiling on the real-time priority that may be
  set for this process.
  \item \texttt{RLIMIT\_RTTIME} -- Limit on the amount of CPU time that a
  process scheduled under a real-time scheduling policy may consume without making a blocking system call.
  \item \texttt{RLIMIT\_SIGPENDING} -- Limit on the number of signals that may
  be queued by the user of the calling process.
  \item \texttt{RLIMIT\_STACK} -- The maximum size of the process stack.
\end{itemize}

For all of the above \texttt{RLIMITs}, the \pkg{RAppArmor} package implements a
function which name is equivalent to the non-capitalized name of the
\texttt{RLIMIT}. For example to get/set \texttt{RLIMIT\_AS}, one can call
\texttt{rlimit\_as()}. Every \texttt{rlimit\_} function has exactly 3 parameters:
\texttt{hardlim}, \texttt{softlim}, and \texttt{pid}. Each argument is
specified as an integer value. The \texttt{pid} arguments points to the target
process. When this argument is omitted, the calling process is targeted. When
the \texttt{softlim} is omitted, it is set equal to the \texttt{hardlim}.

The soft limit is the value that the kernel enforces for the corresponding
resource. The hard limit acts as a ceiling for the soft limit: an unprivileged
process may only set its soft limit to a value in the range from 0 up to the
hard limit, and (irreversibly) lower its hard limit. A  privileged process
(under  Linux:  one  with  the \texttt{CAP\_SYS\_RESOURCE} capability) may make
arbitrary changes to either limit value. When the function is called without any
arguments, it prints the current limits to \texttt{STDOUT}. \citep{linuxrlimit}

\begin{CodeChunk}
\begin{CodeInput}
> library(RAppArmor)
> rlimit_as()
RLIMIT_AS:
Current limits: soft=-1; hard=-1
> A <- rnorm(1e7)
> rm(A)
> gc()
         used (Mb) gc trigger (Mb) max used (Mb)
Ncells 185467  5.0     407500 10.9   350000  9.4
Vcells 176590  1.4    8743611 66.8 10182143 77.7
>
> rlimit_as(10*1024*1024)
RLIMIT_AS:
Previous limits: soft=-1; hard=-1
Current limits: soft=10485760; hard=10485760
> A <- rnorm(1e7)
Error: cannot allocate vector of size 76.3 Mb
\end{CodeInput}
\end{CodeChunk}

Note that a process owned by a user without superuser privileges can only modify
\texttt{RLIMIT} to more restrictive values. However, using \texttt{eval.secure},
a more restrictive \texttt{RLIMIT} can be applied to a single evaluation without
any side effects on the main process:

\begin{CodeChunk}
\begin{CodeInput}
> library(RAppArmor)
> A <- eval.secure(rnorm(1e7), RLIMIT_AS = 10*1024*1024);
Error: cannot allocate vector of size 76.3 Mb
> A <- rnorm(1e7)
\end{CodeInput}
\end{CodeChunk}

The exact meaning of the different limits can be found in the \texttt{RAppArmor}
package documentation (e.g. \texttt{?rlimit\_as}) or in the documentation of
the Linux operating system \citep{ubunturlimit}.

\subsection{Activating AppArmor profiles}

The \pkg{RAppArmor} package implements three calls to the Linux kernel related
to applying AppArmor profiles: \texttt{aa\_change\_profile},
\texttt{aa\_change\_hat} and \texttt{aa\_revert\_hat}. Both the
\texttt{aa\_change\_profile} and \texttt{aa\_change\_hat} functions take a
parameter named \texttt{profile}: a character string identifying the name of
the profile. This profile has to be preloaded by the kernel, before it can be
applied to a process. The easiest way to load profiles is to copy them to the
directory \texttt{/etc/apparmor.d} and then run \texttt{sudo service apparmor
restart}.

The main difference between a \emph{profile} and a \emph{hat} is that switching
profiles is an irreversible action. Once the profile has been associated with
the current process, the process cannot call \texttt{aa\_change\_profile} again
to escape from the profile (that would defeat the purpose). The only exception
to this rule are profiles that contain an explicit \texttt{change\_profile}
directive. The \texttt{aa\_change\_hat} function on the other hand is designed to
associate a process with a security profile in a way that does allows escaping
out of the security profile. In order to realize this, the
\texttt{aa\_change\_hat} takes a second argument called \emph{magic\_token},
which defines a secret key that can be used to \emph{revert} the hat. When
\texttt{aa\_revert\_hat} is called with the same \texttt{magic\_token} that
was used in \texttt{aa\_change\_hat}, the security restrictions are
relieved. This can be very useful in certain applications, however it is also a
security risk. It is important that the code which is running in the sandbox
cannot read the magic token from memory somehow, because then it would be
have a way of breaking out of the sandbox. Hence, storing the
\texttt{magic\_token} in a variable that is readable from within the sandbox
might not be a good idea.

The \pkg{RAppArmor} package ships with a profile called \emph{testprofile} which
contains a hat called \emph{testhat}. We use this profile to demonstrate the
functionality. The profiles have been defined such that \emph{testprofile}
allows access to \texttt{/etc/group} but denies access to \texttt{/etc/passwd}.
The \emph{testhat} denies access to both \texttt{/etc/passwd} and
\texttt{/etc/group}.

\begin{CodeChunk}
\begin{CodeInput}
> library(RAppArmor);
> result <- read.table("/etc/passwd")

> aa_change_profile("testprofile")
Switching profiles...
> passwd <- read.table("/etc/passwd")
Error in file(file, "rt") : cannot open the connection
In addition: Warning message:
In file(file, "rt") : cannot open file '/etc/passwd': Permission denied
> group <- read.table("/etc/group")

> mytoken <- 13337;
> aa_change_hat("testhat", mytoken);
Setting Apparmor Hat...

> passwd <- read.table("/etc/passwd")
Error in file(file, "rt") : cannot open the connection
In addition: Warning message:
In file(file, "rt") : cannot open file '/etc/passwd': Permission denied
> group <- read.table("/etc/group")
Error in file(file, "rt") : cannot open the connection
In addition: Warning message:
In file(file, "rt") : cannot open file '/etc/group': Permission denied

> aa_revert_hat(mytoken);
Reverting AppArmor Hat...

> passwd <- read.table("/etc/passwd")
Error in file(file, "rt") : cannot open the connection
In addition: Warning message:
In file(file, "rt") : cannot open file '/etc/passwd': Permission denied
> group <- read.table("/etc/group")
\end{CodeInput}
\end{CodeChunk}

Just like for \texttt{setuid} and \texttt{rlimit} functions,
\texttt{eval.secure} can be used to enforce an AppArmor security profile on a
single call, witout any side effects. The \texttt{eval.secure} function uses
\texttt{aa\_change\_profile} and is therefore most secure.

\begin{CodeChunk}
\begin{CodeInput}
> out <- eval(read.table("/etc/passwd"))
> nrow(out)
[1] 68
> out <- eval.secure(read.table("/etc/passwd"), profile="testprofile")
Error in file(file, "rt") : cannot open the connection
\end{CodeInput}
\end{CodeChunk}

\subsection{AppArmor without RAppArmor}
\label{usr.bin.r}

The \pkg{RAppArmor} package allows us to dynamically load an AppArmor profile
from within an R session. This gives a great deal of flexability. However, it is
also possible to use AppArmor without the \pkg{RAppArmor} package, by setting a
single profile to be loaded with any running \proglang{R} process.

The RAppArmor package ships with a file named \texttt{usr.bin.r}. At the
installation of the package, this file is copied to \texttt{/etc/apparmor.d/}.
This file is basically a copy of the \texttt{r-user} profile in appendix
\ref{r-user}, however with a small change: where \texttt{r-user} defines
a named profile with 
\begin{verbatim}
  profile r-user {
    ...
  } 
\end{verbatim}
the \texttt{usr.bin.r} file defines a profile specific to a filepath:
\begin{verbatim}
  /usr/bin/R {
    ...
   } 
\end{verbatim}

When using the latter syntax, the profile is automatically associated every time
the file \texttt{/usr/bin/R} is executed (which is the file that runs when
\texttt{R} is entered in the shell). This way we can set some default security
restrictions for our daily work. Profiles tied to a specific program can be
activated by the administrator using:
\begin{verbatim}
  sudo aa-enforce usr.bin.r
\end{verbatim}
This will enforce the security restrictions on every new R process that is
started. To disable the profile, the administrator can run:
\begin{verbatim}
  sudo aa-disable usr.bin.r
\end{verbatim}
After disabling the profile, the R program can be started without any
restrictions. 

Note that the \texttt{usr.bin.r} profile does \textbf{not} grant permission to
change profiles. Hence, one the \texttt{usr.bin.r} profile is in enforce mode,
we cannot use the \texttt{eval.secure} or \texttt{aa\_change\_profile} functions
from the \pkg{RAppArmor} package to change into a different profile, as this
would be a security hole.

\subsection{Learning using Complain Mode}

Finally AppArmor allows the administrator to set profiles in \emph{complain
mode}, which is also called \emph{learning mode}. 
\begin{verbatim}
  sudo aa-complain usr.bin.r
\end{verbatim}
This is useful for developing new profiles. When a profile is set in complain
mode, security restrictions are not actually enforced; instead all violations
of the security policy are logged to the \texttt{syslog} and \texttt{kern.log}
files. This is a powerful way of creaing new profiles: one can set a program in
complain mode during regular use, and afterwards the log files can be used to
study violations of the current policy. From these violations we can determine
which permissions will have to be added to the profile to make the program work
under normal behavior. AppArmor even ships with a powerful utility named
\texttt{aa-logprof} which helps the administrator by parsing log files and
suggesting new rules to be added to the profile. This is a nice way of
debugging a profile, and figuring out which permissions exactly a program
requires to do its work.

\section[Profiling R]{Profiling \proglang{R}: Defining Security Policies}

The ``hard'' part of the problem is actually profiling \proglang{R}. With
profiling we mean defining the policies: which files and directories should
\proglang{R} be allowed to read and write to? Which external programs is it
allowed to execute? Which libraries or shared modules it allowed to load, etc.
We want to minimize ways in which the process could potentially damage the
system, but we don't want to be overly restrictive either: preferebly, users
should be able to do anything they normally do in \proglang{R}. Because R is
such a complete system with a big codebase and a wide range of functionality,
the base system actually already requires quite a lot of access to the file
system.

As often, there is no ``one size fits all'' solution. Depending on which
functionality is needed for an application we might want to grant or deny
certain privileges. We might even want to execute some parts of a process with
tighter privileges than other parts. For example, within a web service, the
service process should be able to write to system log files, which should not be
writable by custom code from a user. We might also want to be more strict on
some users than others, e.g. allow all users to execute code, but only allow
privileged users to install a new package.

\subsection{AppArmor Policy Configuration Syntax}
\label{syntax}

The \emph{AppArmor policy configuration syntax} is used to define the access
control profiles in \texttt{AppArmor}. Other mandatory access control systems
might implement different functionality and require other syntax, but in the end
they address mostly similar issues. AppArmor is a quite advanced and provides
access control over many of the features and resources found in the Linux
kernel, e.g. file access, network rules, Linux capability modes, mounting
rules, etc. All of these can be useful, but most of them are very application
specific. Furthermore, the policy syntax has some additional functionality that
allows for defining \emph{subprofiles}, and \emph{includes}.

The most important form of access control which will be the focus of the
remaining of the section are \emph{file permission access modes}. Once AppArmor
is enforcing mandatory access control, a process can only access files and
directories on the system for which it has explicitly been granted access in
its security profile. Because in \texttt{Linux} almost everything is a file
(even sockets, devices, etc) this gives a great deal of control. AppArmor
defines a number of access modes on files and directories, of which the most
important ones are:

\begin{itemize}
  \item[] \texttt{r} -- read file or directory
  \item[] \texttt{w} -- write to file or directory
  \item[] \texttt{m} -- load file in memory
  \item[] \texttt{px} -- discrete profile execute of executable file
  \item[] \texttt{cs} -- transition to subprofile for executing a file
  \item[] \texttt{ix} -- inherit current profile for executing a file
  \item[] \texttt{ux} -- unconfined execution of executable file (dangerous)
\end{itemize}

Using this syntax we will present some example profiles for \proglang{R}.
Because the profile depends on the absolute locations of system files, we will
assume the standard file layout for Debian and Ubuntu systems. This includes
files that are part of \texttt{r-base} and other packages that are used by
\proglang{R}, e.g. \texttt{texlive}, \texttt{libxml2}, \texttt{bash},
\texttt{libpango}, \texttt{libcairo}, etc.

\subsection[Profile: r-base]{Profile: \texttt{r-base}}

Appendix \ref{r-base} contains a profile that we have named \texttt{r-base}.
It is a fairly basic and general profile. It grants read/load access to all
files in common shared system directories, e.g. \texttt{/usr/lib,
/usr/local/lib, /usr/share}, etc. However, the default profile only grants
write access inside \texttt{/tmp}, not in e.g. the home directory. Furhermore,
\proglang{R} is allowed to execute any of the shell commands in \texttt{/bin}
or \texttt{/usr/bin} for which the program will inherit the restrictions.

\begin{CodeChunk}
\begin{CodeInput}
> library(RAppArmor)
> aa_change_profile("r-base")
Switching profiles...

> #These operations will be denied:
> list.files("/")
character(0)
> list.files("~")
character(0)
> file.create("~/test")
[1] FALSE
> list.files("/tmp")
character(0)
> install.packages("wordcloud")
Error opening file for reading: Permission denied

> #These operations are permitted:
> library(ggplot2);
> setwd(tempdir())
> pdf("test.pdf")
> qplot(speed, dist, data=cars);
> dev.off()
null device
          1
> list.files()
[1] "downloaded_packages"
[2] "libloc_107_669a3e12.rds"
[3] "libloc_118_46fd5f8e.rds"
[4] "libloc_128_97f33314.rds"
[5] "pdf6d1117f7d683"
[6] "repos_http%3a%2f%2fcran.stat.ucla.edu%2fsrc%2fcontrib.rds"
[7] "test.pdf"
> file.remove("test.pdf")
[1] TRUE
\end{CodeInput}
\end{CodeChunk}

The \texttt{r-base} profile effectively prevents R from most malicious
activity, while still allowing access to all of the libraries, fonts, icons, and
programs that it might want to use. One thing to note is that the profile
does not allow listing of the contents of \texttt{/tmp}, but it does allow full
rw access on any of its subdirectories. This is to prevent one process from
reading/writing files from the temp directory of another active R process (given
that it cannot discover the name of the other temp directory).

The \texttt{r-base} profile is a quite liberal and general purpose profile. When
using AppArmor in a more specific application, it is recommended to make the
profile a bit more restrictive by specifying exactly \emph{which} of the
packages, shell commands and system libraries should be accessible by the
application. That could prevent potential problems when vulnerabilities are
found in some of the standard libraries.

\subsection[Profile: r-compile]{Profile: \texttt{r-compile}}

The \texttt{base-r} profile does not allow access to the compiler, nor does it
allow for loading (\texttt{m}) or execution (\texttt{ix}) of files in places
where it can also write. If we want user to be able to compile e.g.
\proglang{C++} code, we will need to give it access to the compiler. In order
to do so, we need to add these lines:

\begin{verbatim}
/usr/include/** r,
/usr/lib/gcc/** rix,
/tmp/** rmw,
\end{verbatim}

Note especially the last line. The \texttt{m} allows \proglang{R} to load shared
objects into memory from anywhere under \texttt{/tmp}. This is needed to load
the compiled code after it has been installed to a temporary directory. Note
that this does not come without a cost: compiled code can potentially contain
malicious code or even exploits that can do harm when loaded into memory. If
this privilege is not needed, it is generally recommended to only allow
\texttt{m} and {ix} access modes on files that have been installed by the
system administrator. The new profile including these rules ships with the
package as \texttt{r-compile} and is also printed in appendix \ref{r-compile}.

After adding the lines above and reloading the profile, it should be possible to
compile a package that contains \proglang{C++} code and install it to somewhere
in \texttt{/tmp}:

\begin{CodeChunk}
\begin{CodeInput}
> eval.secure(install.packages("wordcloud", lib=tempdir()), profile="r-compile");
trying URL 'http://cran.stat.ucla.edu/src/contrib/wordcloud_2.0.tar.gz'
downloaded 36 Kb

* installing *source* package 'wordcloud' ...
** package 'wordcloud' successfully unpacked and MD5 sums checked
** libs
g++ -I/usr/share/R/include -DNDEBUG -I"/usr/local/lib/R/site-library/Rcpp/include"
   -fpic  -O3 -pipe  -g  -c layout.cpp -o layout.o
g++ -shared -o wordcloud.so layout.o -L/usr/local/lib/R/site-library/Rcpp/lib
   -lRcpp -Wl,-rpath,/usr/local/lib/R/site-library/Rcpp/lib -L/usr/lib/R/lib -lR
installing to /tmp/RtmpFCM6WS/wordcloud/libs
** R
** data
** preparing package for lazy loading
** help
*** installing help indices
** building package indices
** testing if installed package can be loaded

* DONE (wordcloud)

The downloaded source packages are in
	'/tmp/RtmpFCM6WS/downloaded_packages'
\end{CodeInput}
\end{CodeChunk}

\subsection[Profile: r-user]{Profile: \texttt{r-user}}

Appendix \ref{r-user} defines a profile named \texttt{r-user}. This profile is
designed to be a nice balance between security and freedom for day to day use of
R. It extends the \texttt{r-compile} profile with some additional privileges
with respect to the users home directory. The variable \texttt{@\{HOME\}} is
defined in the \texttt{tunables/global} include and matches the location of the
users home directory, i.e. \texttt{/home/jeroen} for a user named ``jeroen''.
The profile assumes that there is a directory named \texttt{R} directly inside
the home directory (e.g \texttt{/home/jeroen/R}), to which R can read and
write. Furthermore, R can load and execute files in the directories
\texttt{i686-pc-linux-gnu-library} and \texttt{x86\_64-pc-linux-gnu-library}
inside of this \texttt{R} directory. These are the standard locations where
\texttt{R} installs a users personal package library.

Using the \texttt{r-user} profile, the user will be able to do most of his day
to day work, including installing and loading new packages in his personal
library, while still being protected against most malicious activities. The
\texttt{r-user} profile is also the basis of the default \texttt{usr.bin.r}
profile as described in section \ref{usr.bin.r}.

\subsection{Installing packages}

An additional privilege that might be needed in some situations is the option to
install packages to the machines global library, which is readable by all users.
In order to allow this, a profile needs to include write access to the
\texttt{site-library} directory:

\begin{verbatim}
/usr/local/lib/R/site-library rmw,
\end{verbatim}

After adding this line to a profile, the policy will allow for installing R
packages to the global site library. However, note that AppArmor does not
replace, but \emph{supplements} the standard access control system. Hence if a
user does not have superuser privileges it will still not be able to install
packages in the global site library, even though the AppArmor profile does grant
this permission.

\newpage
\section{Concluding remarks}

In this paper the reader was introduced to some potential security issues
related to the use of the \proglang{R} software. We hope to have raised some
awareness that security is an increasingly important concern for the
\proglang{R} user, but also that addressing this issue could open doors to new
applications of the software. The \pkg{RAppArmor} package was introduced as an
example that demonstrates how some security issues could be addressed on
\texttt{Linux} systems.

Our implementation should prevent malicious activity, but this is just one way
to do it. Another method that might be interesting is provided by 
\texttt{Linux CGroups}. These could provide a finer degree of control of allocation and security using
hierarchical process groups. A completely different direction might be offered
by \texttt{renjin} \citep{renjin}, a JVM-based Interpreter for the R Language.
If R code can be executed though the \texttt{JVM}, we might be able to use some
tools from the \texttt{Java} community to solve the same problems.

However, no matter which tools are used, security profiling always comes down to
balancing \emph{regular use} and \emph{malicious use}. This has a big human aspect to it, and is can be a
learning process in itself. It might not be until you actually put an application in
production that you start getting complains from users that their favorite
package is not working, or that you find out that some users are abusing the
system in a way that you could not have foreseen. 

Furthermore, security has to be implemented in a way that is practical. When the
security policies disrupt the working flow or make it imposible to get anything
done, most users will decide to not use any security at all. 


\newpage

\begin{appendices}
\section{Example Profiles}

This appendix prints some of the example profiles that ship with the
\pkg{RAppArmor} package. To load them in AppArmor, simply copy-paste the
profile into a file that you put in the directory \texttt{/etc/apparmor.d} and
then run \texttt{sudo service apparmor restart}. You should then be able to load
them into an \proglang{R} session using either the \texttt{aa\_change\_profile}
or \texttt{secure.eval} function from the \texttt{RAppArmor} package.

\subsection[Profile: r-base]{Profile: \texttt{r-base}}
\label{r-base}

\begin{verbatim}
#include <tunables/global>
profile r-base {
        #include <abstractions/base>
        #include <abstractions/nameservice>

        /bin/* rix,
        /etc/R/ r,
        /etc/R/* r,
        /etc/fonts/** mr,
        /etc/resolv.conf r,
        /etc/xml/* r,
        /tmp/** rw,
        /usr/bin/* rix,
        /usr/lib/R/bin/* rix,
        /usr/lib{,32,64}/** mr,
        /usr/lib{,32,64}/R/bin/exec/R rix,
        /usr/local/lib/R/** mr,
        /usr/local/share/** mr,
        /usr/share/** mr,
        /usr/share/ca-certificates/** r,
}
\end{verbatim}


\subsection[Profile: r-compile]{Profile: \texttt{r-compile}}
\label{r-compile}

\begin{verbatim}
#include <tunables/global>
profile r-compile {
        #include <abstractions/base>
        #include <abstractions/nameservice>

        /bin/* rix,
        /etc/R/ r,
        /etc/R/* r,
        /etc/fonts/** mr,
        /etc/resolv.conf r,
        /etc/xml/* r,
        /tmp/** rmw,    
        /usr/bin/* rix,
        /usr/include/** r,       
        /usr/lib/gcc/** rix,		 
        /usr/lib/R/bin/* rix,
        /usr/lib{,32,64}/** mr,
        /usr/lib{,32,64}/R/bin/exec/R rix,
        /usr/local/lib/R/** mr,
        /usr/local/share/** mr,
        /usr/share/** mr,
        /usr/share/ca-certificates/** r,
}
\end{verbatim}

\subsection[Profile: r-user]{Profile: \texttt{r-user}}
\label{r-user}

\begin{verbatim}
#include <tunables/global>
profile r-user {
        #include <abstractions/base>
        #include <abstractions/nameservice>
	
        capability kill,
        capability net_bind_service,
        capability sys_tty_config,
	
        @{HOME}/ r,
        @{HOME}/R/ r,
        @{HOME}/R/** rw,
        @{HOME}/R/{i686,x86_64}-pc-linux-gnu-library/** mrwix,
        /bin/* rix,
        /etc/R/ r,
        /etc/R/* r,
        /etc/fonts/** mr,
        /etc/resolv.conf r,
        /etc/xml/* r,
        /tmp/** mrwix,
        /usr/bin/* rix,
        /usr/include/** r,       
        /usr/lib/gcc/** rix,		
        /usr/lib/R/bin/* rix,
        /usr/lib{,32,64}/** mr,
        /usr/lib{,32,64}/R/bin/exec/R rix,
        /usr/local/lib/R/** mr,
        /usr/local/share/** mr,
        /usr/share/** mr,
        /usr/share/ca-certificates/** r,
}
\end{verbatim}



\section{Security Unit Tests}

This appendix prints a number of unit tests that contain malicious code and
which should be prevented by any sandboxing tool.

\subsection{Access System Files}

Usually R has no business in the system logs, and these are not included in the
profiles. The codechunk below attempts to read the syslog file.
\begin{CodeChunk}
\begin{CodeInput}
readSyslog <- function(){
	readLines('/var/log/syslog');
}
\end{CodeInput}
\end{CodeChunk}
When executing this with a r-user profile, access to this file is denied,
resulting in an error:
\begin{CodeChunk}
\begin{CodeInput}
> eval.secure(readSyslog(), profile='r-user')
Switching profiles...
Error in file(con, "r") : cannot open the connection
\end{CodeInput}
\end{CodeChunk}

\subsection{Access User Files}
\label{creditcard}

Access to system files can to some extend by prevented by running processes as
non privileged users. But it is easy to forget that also the users personal
files can contain senstive information. Below a simple function that scans the
\texttt{Documents} directory of the current user for files that contain credit
card numbers. 

\begin{CodeChunk}
\begin{CodeInput}
findCreditCards <- function(){
  pattern <- "([0-9]{4}[- ]){3}[0-9]{4}"
  for (filename in list.files("~/Documents", full.names=TRUE, recursive=TRUE)){
    if(file.info(filename)$size > 1e6) next;
    doc <- readLines(filename)
    results <- gregexpr(pattern, doc)
    output <- unlist(regmatches(doc, results));
    if(length(output) > 0){
      cat(paste(filename, ":", output, collapse="\n"), "\n")
    }
  }
}
\end{CodeInput}
\end{CodeChunk}

This example prints the credit card numbers to the user, but it would be quite
easy to post them to some server on the internet. For this reason the
\texttt{r-user} profile denies access to the users home dir, except for the 
\texttt{{\raise.17ex\hbox{$\scriptstyle\sim$}}/R} directory.


\subsection{Limit Memory}

When a system or service is used by many users at the same time, it is important
that we cap the memory that can be used by a single process. The following
function generates a quite large matrix:

\begin{CodeChunk}
\begin{CodeInput}
memtest <- function(){
	A <- matrix(rnorm(1e7), 1e4);
}
\end{CodeInput}
\end{CodeChunk}

When R tries to allocate more memory than allowed, it will throw an error:

\begin{CodeChunk}
\begin{CodeInput}
> A <- eval.secure(memtest(), RLIMIT_AS = 1000*1024*1024)
RLIMIT_AS:
Previous limits: soft=-1; hard=-1
Current limits: soft=1048576000; hard=1048576000
> rm(A)
> gc()
         used (Mb) gc trigger  (Mb) max used  (Mb)
Ncells 193074 10.4     407500  21.8   350000  18.7
Vcells 299822  2.3   17248096 131.6 20301001 154.9


> A <- eval.secure(memtest(), RLIMIT_AS = 100*1024*1024)
RLIMIT_AS:
Previous limits: soft=-1; hard=-1
Current limits: soft=104857600; hard=104857600
Error: cannot allocate vector of size 76.3 Mb
\end{CodeInput}
\end{CodeChunk}


\subsection{Limit CPU Time}

Suppose we are hosting a web service and we want to kill jobs that do not finish
in 5 seconds. Below a snippet that will take much more than 5 seconds on most
machines. Note that because R calling out to \texttt{C} code, it will not be
possible to terminate this function prematurely using R's \texttt{setTimeLimit}
or even using \texttt{CTRL+C} in an interactive console. If this would happen
inside of a bigger system, the entire service might become unresponsive.

\begin{CodeChunk}
\begin{CodeInput}
cputest <- function(){
  A <- matrix(rnorm(1e7), 1e3);
  B <- svd(A);
}
\end{CodeInput}
\end{CodeChunk}
In RAppArmor we have actuall two different options to deal with this. The first
one is setting the \texttt{RLIMIT\_CPU} value. This will cause the kernel to
kill the process after 5 seconds: 
\begin{CodeChunk}
\begin{CodeInput}
> eval.secure(cputest(), RLIMIT_CPU=5)
RLIMIT_CPU:
Previous limits: soft=-1; hard=-1
Current limits: soft=5; hard=5

NULL
> Sys.time()
[1] "2012-07-08 17:08:35 CEST"
>
\end{CodeInput}
\end{CodeChunk}
However, this is actually a bit rough: because the kernel actually terminates
the process after 5 seconds we have no control over what should happen, nor can
we throw an informative error. A more elegant solution is to kill the process
ourselves using the \texttt{timeout} argument from the \texttt{eval.secure}
function. Because the actual job is processed in a fork, the parent process
stays responsive, and is used to kill the child process.
\begin{CodeChunk}
\begin{CodeInput}
> Sys.time()
[1] "2012-07-08 16:59:06 CEST"
> eval.secure(cputest(), timeout=5)
Error: R call did not return within 5 seconds. Terminating process.
> Sys.time()
[1] "2012-07-08 16:59:11 CEST"
\end{CodeInput}
\end{CodeChunk}

\subsection{Fork Bomb}

A fork bomb is a process that spawns many child processes, which often results
in the operating system getting stuck to a point where it has to be rebooted.
Doing a fork bomb in R is quite easy and requires no special privileges:
\begin{CodeChunk}
\begin{CodeInput}
forkbomb <- function(){
  repeat{
    parallel::mcparallel(forkbomb());
  }
}
\end{CodeInput}
\end{CodeChunk}
Do not run call this function outside sandbox, because it will make the machine
unresponsive. However, inside our sandbox we can use the \texttt{RLIMIT\_NPROC}
to limit the number of processes the user is allowed to own:
\begin{CodeChunk}
\begin{CodeInput}
> eval.secure(forkbomb(), RLIMIT_NPROC = 20)
RLIMIT_NPROC:
Previous limits: soft=39048; hard=39048
Current limits: soft=20; hard=20
Error in mcfork() : 
  unable to fork, possible reason: Resource temporarily unavailable
\end{CodeInput}
\end{CodeChunk}
Note that the process count is based on the Linux user. Hence if the same Linux
user already has a number of other processes, which is usually the case for
non-system users, the cap has to be higher than this number. Different
processes owned by a single user can enforce different \texttt{NPROC} limits,
however in the actual process count all active processes from the current user
are taken into account. Therefore it might make sense to create a separate
Linux system user that is only used to process R jobs. That way \texttt{RLIMIT\_NPROC} actually
corresponds to the number of \proglang{R} processes. The \texttt{eval.secure}
function has arguments \texttt{uid} and \texttt{gid} that can be used to switch
Linux users before evaluating the call.


\end{appendices}

\newpage

%\bibliographystyle{apalike-url}	% (uses file "plain.bst")
\bibliography{document}

\end{document}
